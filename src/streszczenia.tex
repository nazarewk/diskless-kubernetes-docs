\newpage
\begin{center}
  \large \bf
  Implementacja środowiska Kubernetes na maszynach bezdyskowych
\end{center}

\section*{Streszczenie}

Celem pracy jest przegląd zagadnień związanych z systemem Kubernetes oraz
przykładowa implementacja klastra na maszynach bezdyskowych.

W pierwszej części przedstawione zostały pojęcia kontenerów, wymienione
dostępne narzędzia zarządzania nimi oraz

W drugiej części przedstawione zostały: Kubernetes, jego architektura i podstawowe
zagadnienia pozwalające na zrozumienie i korzystanie z niego.
Rozdział zakończy się teoretycznym przeglądem narzędzi konfigurujących klaster
Kubernetes na sprzęcie bezdyskowym.

TODO: NAPISAC TO

Po ich wybraniu przeprowadzę testy na sieci uczelnianej, a na koniec doprowadzę
ją do stanu docelowego pozwalającego na przeprowadzenie laboratoriów Kubernetes.

\bigskip
{\noindent\bf Słowa kluczowe:} Kubernetes, konteneryzacja, zarządzanie kontenerami, maszyny bezdyskowe

\vskip 2cm


\begin{center}
  \large \bf
  Implementing Kubernetes on diskless machines
\end{center}

\section*{Abstract}


\bigskip
{\noindent\bf Keywords:} Kubernetes, containerization, container orchestration, diskless systems

TODO: WRITE THIS

\vfill
