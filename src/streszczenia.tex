\newpage
\begin{center}
  \large \bf
  Implementacja środowiska Kubernetes na maszynach bezdyskowych
\end{center}

\section*{Streszczenie}

Celem tej pracy inżynierskiej jest przybliżenie czytelnikowi zagadnień
związanych z systemem Kubernetes oraz jego uruchamianiem na maszynach
bezdyskowych.

Zacznę od wyjaśnienia pojęcia kontenerów, problemu orkiestracji nimi i krótkiego
teoretycznego przeglądu dostępnych rozwiązań.
Opiszę czym jest Kubernetes, jaką ma architekturę oraz przedstawię podstawowe
pojęcia pozwalające na zrozumienie i korzystanie z niego.
Opis Kubernetes zakończę przedstawieniem sposobów jego uruchomienia na maszynach
bezdyskowych.

Następnie przeprowadzę krótki teoretyczno-praktyczny przegląd systemów
operacyjnych i sposobów uruchamiania Kubernetes na nich.

Po ich wybraniu przeprowadzę testy na sieci uczelnianej, a na koniec doprowadzę
ją do stanu docelowego pozwalającego na przeprowadzenie laboratoriów Kubernetes.

\bigskip
{\noindent\bf Słowa kluczowe:} Kubernetes, konteneryzacja, orkiestracja, maszyny bezdyskowe

\vskip 2cm


\begin{center}
  \large \bf
  Implementing Kubernetes on diskless machines
\end{center}

\section*{Abstract}

Primary goal of this document is to present basic concepts related to Kuberentes
and running it on diskless systems.

I will start with explaining what are containers, problem of their orchestration
and I will theoretically inspect available solutions.

I will describe what is Kuberentes, provide overview of it's architecture and
basic concepts allowing to understand and use it.
I will end the description with overview of it's provisioning tools working on
diskless systems.

Then I will research and conduct brief practical tests of available operating
systems and provisioning tools.

After selecting solutions I will conduct practical tests on university network
and finally configure Kubernetes to run the network.

\bigskip
{\noindent\bf Keywords:} Kubernetes, containerization, orchestration, diskless systems

\vfill