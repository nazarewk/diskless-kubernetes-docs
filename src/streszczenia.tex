\newpage
\begin{center}
  \large \bf
  Implementacja i testy wydajnosci środowiska Kubernetes na maszynach bezdyskowych
\end{center}

\section*{Streszczenie}

Celem tej pracy inżynierskiej jest przybliżenie czytelnikowi zagadnień
związanych z uruchamianiem systemu Kubernetes na maszynach bezdyskowych.

Zacznę od wyjaśnienia pojęcia systemu bezdyskowego oraz sposobu jego
funkcjonowania na przykładzie sieci uczelnianej i wzorującego się na niej
przygotowanie przeze mnie lokalnego środowiska.

Następnie opiszę problem izolacji i przydzielania zasobów systemowych
na przykładzie wirtualnych maszyn, chroot i konteneryzacji.

W głównej części dokumentu przedstawię pojęcie orkiestrami kontenerami,
w jaki sposób odnosi się do wcześniej postawionych problemów. Opiszę
alternatywy Kubernetes, jego architekturę oraz sposoby uruchamiania.
Na koniec spróbuję uruchomić Kubernetes na maszynach bezdyskowych,
problemy z tym związane oraz przedstawię wyniki.

\bigskip
{\noindent\bf Słowa kluczowe:} praca dyplomowa, LaTeX, jakość

\vskip 2cm


\begin{center}
  \large \bf
  Implementing and testing Kubernetes running on diskless machines
\end{center}

\section*{Abstract}

Lorem ipsum dolor sit amet, consectetur adipiscing elit.
Morbi ac dolor scelerisque, malesuada ex vel, feugiat augue.
Suspendisse dictum, elit efficitur vestibulum eleifend, mi neque
accumsan velit, at ultricies ex lectus et urna. Pellentesque vel
lorem turpis. Donec blandit arcu lacus, vitae dapibus tellus tempus et.
Etiam orci libero, mollis in dapibus tempor, rutrum eget magna.
Nullam congue libero non velit suscipit, vel cursus elit commodo.
Praesent mollis augue quis lorem laoreet, condimentum scelerisque ex pharetra.
Sed est ex, gravida a porta in, tristique ac nunc. Nunc at varius sem, sit amet consectetur velit.

\bigskip
{\noindent\bf Keywords:} thesis, LaTeX, quality

\vfill