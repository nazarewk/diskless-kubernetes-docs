\newpage
\begin{center}
  \large \bf
  Implementacja i testy wydajności środowiska Kubernetes na maszynach bezdyskowych
\end{center}

\section*{Streszczenie}

Celem pracy jest implementacja klastra Kubernetes w sieci komputerowej bazującej
na maszynach bezdyskowych.
W pracy przedstawione zostały podstawowe pojęcia związane z kontenerami,
systemem Kubernetes i maszynami bezdyskowymi.
Część praktyczna zawiera przegląd systemów operacyjnych i dostępnych
konfiguratorów klastra Kubernetes, a kończy się prezentacją implementacji
klastra Kubernetes w sieci komputerowej
Wydziału Elektrycznego Politechniki Warszawskiej.

\bigskip
{\noindent\bf Słowa kluczowe:} Kubernetes, konteneryzacja, zarządzanie kontenerami, systemy bezdyskowe

\vskip 2cm


\begin{center}
  \large \bf
  Implementing cluster Kubernetes based on diskless machines
\end{center}

\section*{Abstract}

Purpose of the this thesis is implementation of Kubernetes cluster in
the network based on diskless systems.
The paper presents basic concepts of containers, Kubernetes system and
diskless machines.
This paper includes practical overview of operating systems and available
Kubernetes configurators. Publication ends with a presentation of
real world deployment of Kubernetes cluster in the
Faculty of Electrical Engineering of Warsaw University of Technology
network.

\bigskip
{\noindent\bf Keywords:} Kubernetes, containerization, container orchestration, diskless systems

\vfill
